	\subsection{Ten-Week Term Retrospective}
		\begin{longtable}{|l|p{0.3\linewidth}|p{0.3\linewidth}|p{0.3\linewidth}|}\hline \textbf{Week} & \textbf{Positives} & \textbf{Deltas} & \textbf{Actions}\\\hline
		1-Jan 	& - & - & -\\\hline

		2-Jan 	&
\begin{itemize}
\item back and forth email communication with Kevin regarding to way to get into the CS 462 and CS 406 
\end{itemize}
			&
\begin{itemize}
\item Kevin emailed me back regarding the way i wanted to work in a new project: to whether i have decided to work with a partner or to do it in solo.
\item The next day exactly on Friday, January the 19th he informed me that he is making an arrangement.
\end{itemize}
			&
\begin{itemize}
\item I emailed Kevin to know what would the next step after registering to the cs 406 and cs 462.
\item I replied to Kevin that " working with a partner would be beneficial to me and allow to simulate   a real-world experience.
\end{itemize}

			\\\hline

		3-Jan	&
\begin{itemize}
\item This week, the Group 69 is born
\item For the first time I had a fruitful exchange with my partner Lam
 
\end{itemize}
			&
\begin{itemize}
\item At this point, I did not know to jump onboard and get started with the project though i could note a positive interaction with Lam.
\end{itemize}
			&
\begin{itemize}
\item i reached Lam out to get the insight about the new project and for the first amazingly, the promptly respond to my email by outlining the key points of the Depht sensing with computer vision and lidar project, which at that times was titled ROS lidar project.
\item I was encouraged by Lam to reach out Junki as he was introduced to us by Kevin as our  TA and ultimately Junki accepted to be our TA.
 
\end{itemize}
			\\\hline

		4-Jan	&
\begin{itemize}
\item if week 3 was the week that our group was born, week 4 actually indicates when i effectively got started on searching what could be one way of solving the DSCVL problem.
\end{itemize}
			&
\begin{itemize}
\item I got into my first struggle trying to install the python libray on Windows 10
\item I had no idea how to get the rplidar library worked on Windows.  

\end{itemize}
			&
	\textbf{ weekly tasks list}  	
\begin{enumerate}
\item Install RPLiar python library [5]
\item Build the appJar  file from the appJar library for the GUI [3] 
\end{enumerate}
			\\\hline

		5-Fev 	&
\begin{itemize}
\item noticeable progress done by building python libraries and configuring paths to properly work in windows environment 
\end{itemize}
			&
\begin{itemize}
\item I met with Kevin during the weekly office hour session and put forward an sketched figure of how I was designing  final overlay image and also how would the relative distance of the object to be captured by the Robotpeak lidar device. 
\end{itemize}
			&
	\textbf{ weekly tasks list}
	\begin{enumerate}
	\item Download and install Numpy  wheel files
	\item Download and install OpenCV wheel file
	\item Configure " pip" tool to run with the Windows 10
	\item explore how to work with tkinter library  
	\end{enumerate}

			\\\hline

		6-Fev	&
\begin{itemize}
\item week for the midterm progress report write up
\item built the pdf slide presentation file along with its video
\end{itemize}
			&
\begin{itemize}
\item 
\end{itemize}
			&
			\textbf{ weekly tasks list}
\begin{enumerate}
\item 	Turn in midterm process report ( slide, write up, audio recording)
\item wrote the first function to capture the video stream from the my built-in webcam and store images on the Hard drive.
\end{enumerate}
			\\\hline

		7-Fev	&
\begin{itemize}
\item explore other possibilities to build a virtual Linux layer on top of the windows OS.
\item work with the Group 40 learder to have a private critique poster session
\end{itemize}
			&
\begin{itemize}
\item Faced a serious setbadk to make the windows 10 system work with the rplidar device. 
\item I was having buggy results regardless changes made in the source code program
\end{itemize}
			&
			\textbf{ weekly tasks list}
\begin{enumerate}

\item Read documentation to perform test reading from the lidar device
\item Helped out by Kevin to successfully read data from the  lidar A1 scanner
\end{enumerate}

			\\\hline

		8-Fev	&
\begin{itemize}
\item we missed the class poster critique session, and planned to attend the one with Group 40.
\end{itemize}
			&
\begin{itemize}
\item 
\end{itemize}
			&
			\textbf{ weekly tasks list}
\begin{enumerate}
\item raised questions that prompted me to go back and take a closer look at how the rplidar A1  is trying to read data and how i can utilize those data for Z-axis calculations.
\end{enumerate}
			
			\\\hline

		9-March	&
		\begin{itemize}
		\item Poster Critique session meeting in Kelly Hall with Kirsten as supervisor,  Group 40 team.
		\item outcomes of the session/ poster format correction: the need of labeling the central image with axis was presented, Include group members contact  information if necessary and include the client of the project.
		\item explored some the benefits of the Tensorflow technology and in particular the "eager execution" version   
		\end{itemize}

			&
\begin{itemize}
\item  I Was told by Junki and Lam that Tensorflow will not be helpful for the z-axis calculus.
\end{itemize}
			&
			
			\textbf{ weekly tasks list}
\begin{enumerate}
\item contact Nathan leader of Group 40 for the final arrangement for the critique session
\item loop back to update my team partner Lam abot the critique session.
\item i successfully wrote functions to turn rplidar stream data into an array of float values. And then, for that array of float points to be array of values to represent the mobility of the object in front of the lidar device.
\item Install of the eager execution python wheel file on windows 10 
\end{enumerate}
			\\\hline

		10-March	&
\begin{itemize}
\item compute the z-axis object movements using geometrical maths functions
\item explore how i would be using the numpy spacial functions to get around with the standard maths functions [6]
\end{itemize}
			&
\begin{itemize}
\item Still struggling to transform the array float points of data into 2-D array to pass it as one of arguments for the  numpy function.  
\end{itemize}			
			
			&
			\textbf{ weekly tasks list}
	\begin{enumerate}
	\item  Install spicy spacial numpy wheel file on windows 10
	\item write the final winter progress report 
	\item write the slide of the report presentation and build the video
	\
	\end{enumerate}

			\\\hline
			
					11-Apr	&
\begin{itemize}
\item read  the python documentation for C extension
\item explore the possibility to use command line for module extension
\end{itemize}
			&
\begin{itemize}
\item still unable to clearly understand how   C can be extended in python.  
\end{itemize}			
			
			&
			\textbf{ weekly tasks list}
	\begin{enumerate}
	\item  search more info in the python documentation
	\item  find information about static libray 
	\end{enumerate}

			\\\hline
			
					12-Apr	&
\begin{itemize}
\item Find out website to provide samples of project where dynamic Library C/python is used in command lines.
\end{itemize}
			&
\begin{itemize}
\item Still do not know how to interpret  data form the M16 documentation.  
\end{itemize}			
			
			&
			\textbf{ weekly tasks list}
	\begin{enumerate}
	\item  Download and Install of the Leddar M16 exe file for Windows 10 (64 bits)
	\item Configuration of the M16 in Windows 10 environment 
	\item Import the Ctypes function library in python  
	\
	\end{enumerate}

			\\\hline
			
					13-Apr	&
\begin{itemize}
\item commands to Create a dynamic library to link files in Linux
\item  commands how to compile dynamic library in Linux 
\end{itemize}
			&
\begin{itemize}
\item    keep searching how to implement DLL in Windows  
\end{itemize}			
			
			&
			\textbf{ weekly tasks list}
	\begin{enumerate}
	\item  save all the linux commands for DLL
	\item  sample command:  and gcc -shared -o  lnkfile.so file1.c file2.c  ..filex.c
	\end{enumerate}

			\\\hline
			
					14-Apr	&
\begin{itemize}
\item Get the main website describing how to create C extension in python
 
\end{itemize}
			&
\begin{itemize}
\item confront to the problem of Visual Studio.  
\end{itemize}			
			
			&
			\textbf{ weekly tasks list}
	\begin{enumerate}
	\item  Install Visual Studio But computer freezes 
	\item Perform some extra work on the computer to keep on with the project
	
	\end{enumerate}

			\\\hline
			
					15-May	&
\begin{itemize}
\item 
\item Explore other possibilties to call Leddar M16 function.
\end{itemize}
			&
\begin{itemize}
\item  Erreur prompt by visual studio on the C function called
\end{itemize}			
			
			&
			\textbf{ weekly tasks list}
	\begin{enumerate}
	\item  determine what function C to run on API  
	\item Define of the module python on visual studio 2017  
	\item document the first process to call function in Visual Studio
	\
	\end{enumerate}

			\\\hline
			
					16-May	&
\begin{itemize}
\item  examine how to include and link SDK files in Visual Studio (VS)
\end{itemize}
			&
\begin{itemize}
\item  Still facing failure issue at the python/C API and VS  
\end{itemize}			
			
			&
			\textbf{ weekly tasks list}
	\begin{enumerate}
	\item  Headers importation in VS
	\item  Write some main function in the API module
	 
	\end{enumerate}

			\\\hline
			
					17-May	&
\begin{itemize}
\item  explore other paths of writing the API 
\end{itemize}
			&
\begin{itemize}
\item drawback, the Seasnake  implemenmtation could not work in VS 2017
\end{itemize}			
			
			&
			\textbf{ weekly tasks list}
	\begin{enumerate}
	\item Get around the Visual Studio issues of  linkage
	\item Attempt to use seasnake 
	\item document seasnake installation process
	\item Keep working on the previous Python/C API
	\end{enumerate}

			\\\hline
			
					18-May	&
\begin{itemize}
\item  Explore every single possibility to include SDK in VS   
\end{itemize}
			&
\begin{itemize}
\item Struggle with Windows System errors and problems of machine failure.   
\end{itemize}			
			
			&
			\textbf{ weekly tasks list}
	\begin{enumerate}
	\item  Maintain Machine and re-install the environment Visual Studio Windows
	\item reconfigure the entire project
	\item  continue testing integration of the C Leddar in VS
	\item successfully include SDK links in linker
	\end{enumerate}

			\\\hline
			
					19-June	&
\begin{itemize}
\item successfully compile of the ptython/C++ API 
 
\end{itemize}
			&
\begin{itemize}
\item Struggle to call the function Leddar in python code  
\end{itemize}			
			
			&
			\textbf{ weekly tasks list}
	\begin{enumerate}
	\item  Include all headers in the C/API 
	\item Include all files in different modules  
	\item Include links and other files in the VS properties option
	\
	\end{enumerate}

			\\\hline
			
					20-June	&
\begin{itemize}
\item  Narrow the problem of the module python not validating extension at the visual studio level
 
\end{itemize}
			&
\begin{itemize}
\item   Module python still not working  
\end{itemize}			
			
			&
			\textbf{ weekly tasks list}
	\begin{enumerate}
	\item Write final report Presentation 
	\item write final winter progress report 
	\end{enumerate}

			\\\hline
			
		\end{longtable}
